\documentclass[a4paper, 12pt]{article}
\usepackage[margin=2.5cm]{geometry}
\usepackage{amsmath}
\usepackage{amssymb}
\usepackage{graphicx}
\usepackage{enumitem}
\usepackage[indonesian]{babel}
\usepackage{parskip}
\usepackage{tikz}
\usetikzlibrary{decorations.pathreplacing, calc, positioning, arrows.meta}

% Definisi warna untuk bar model yang lebih lembut (pastel) agar tulisan lebih terbaca
\definecolor{barblue}{RGB}{100, 149, 237}
\definecolor{barorange}{RGB}{255, 179, 71}
\definecolor{bargreen}{RGB}{60, 179, 113}
\definecolor{barred}{RGB}{220, 50, 70}
\definecolor{bargray}{RGB}{220, 220, 220}

% Judul Dokumen
\title{\textbf{Pembahasan Soal Matematika Tingkat Lanjut}\\ \textit{(Challenging Word Problems) dengan Pendekatan Bar Model}}
\author{Asisten AI}
\date{}

\begin{document}

\maketitle

\section*{Soal 1: Aljabar (Algebra)}
\textit{Sumber: Halaman 12, Soal 5}

\textbf{Soal:}
Terdapat $x$ permen di dalam sebuah tas. Jane mengambil beberapa permen dari tas tersebut. Ruth mengambil dua kali lipat jumlah permen yang diambil Jane. Sally mengambil 25 permen lebih banyak daripada Jane. Tas tersebut sekarang kosong.
\begin{enumerate}[label=(\alph*)]
    \item Nyatakan jumlah permen yang diambil Jane dalam bentuk $x$.
    \item Jika $x = 265$, tentukan jumlah permen yang diambil oleh masing-masing anak.
\end{enumerate}

\hrulefill

\textbf{Pembahasan dengan Bar Model:}

Mari kita visualisasikan jumlah permen yang diambil setiap anak sebagai blok (unit).
Misalkan jumlah permen Jane adalah 1 unit.

\begin{center}
\begin{tikzpicture}[scale=0.8, every node/.style={font=\small}]
    % Jane
    \node[anchor=east, align=right, text width=2cm] at (-0.2, 3) {Jane:};
    \draw[fill=barblue] (0, 2.5) rectangle (2, 3.5) node[midway, white, font=\bfseries] {1 unit};
    
    % Ruth (2x Jane)
    \node[anchor=east, align=right, text width=2cm] at (-0.2, 1) {Ruth:};
    \draw[fill=barblue] (0, 0.5) rectangle (2, 1.5) node[midway, white, font=\bfseries] {1 unit};
    \draw[fill=barblue] (2, 0.5) rectangle (4, 1.5) node[midway, white, font=\bfseries] {1 unit};
    
    % Sally (Jane + 25) - Geser ke bawah lagi agar tidak rapat
    \node[anchor=east, align=right, text width=2cm] at (-0.2, -1.5) {Sally:};
    \draw[fill=barblue] (0, -2) rectangle (2, -1) node[midway, white, font=\bfseries] {1 unit};
    \draw[fill=barorange] (2, -2) rectangle (3.5, -1) node[midway, white, font=\bfseries] {25};
    
    % Total Brace - Geser lebih ke kanan (x=4.8) agar tidak menabrak bar Ruth
    \draw[decorate,decoration={brace,amplitude=10pt,mirror},thick] (4.8, -2) -- (4.8, 3.5) node[midway, right=0.5cm] {\textbf{Total} $= x$};
    
    % Garis putus-putus bantu visualisasi 1 unit
    \draw[dashed, gray] (2, 3.5) -- (2, -2);
\end{tikzpicture}
\end{center}

Dari model di atas, kita bisa melihat bahwa total permen ($x$) terdiri dari:
\[ \text{4 unit (kotak biru)} + 25 \]

\textbf{(a) Menyatakan Jane dalam $x$}
Karena Jane mewakili 1 unit:
\begin{align*}
    4 \text{ unit} + 25 &= x \\
    4 \text{ unit} &= x - 25 \\
    1 \text{ unit} &= \frac{x - 25}{4}
\end{align*}
Jadi, permen Jane = $\boxed{\frac{x - 25}{4}}$.

\textbf{(b) Jika $x = 265$}
\begin{align*}
    4 \text{ unit} &= 265 - 25 \\
    4 \text{ unit} &= 240 \\
    1 \text{ unit} &= 60
\end{align*}

Maka:
\begin{itemize}
    \item \textbf{Jane} (1 unit) = \textbf{60}
    \item \textbf{Ruth} (2 unit) = $60 \times 2$ = \textbf{120}
    \item \textbf{Sally} (1 unit + 25) = $60 + 25$ = \textbf{85}
\end{itemize}

\newpage

\section*{Soal 2: Perbandingan (Ratio)}
\textit{Sumber: Halaman 51, Soal 5}

\textbf{Soal:}
Kota P memiliki $\frac{2}{5}$ jumlah penduduk Kota Q. Perbandingan jumlah penduduk di Kota R terhadap Kota P adalah $4:7$. Kota Q memiliki 8100 penduduk lebih banyak daripada Kota R. Berapa jumlah penduduk di Kota R?

\hrulefill

\textbf{Pembahasan dengan Bar Model:}

Langkah pertama adalah menyamakan satuan unit untuk Kota P.
\begin{itemize}
    \item P : Q = 2 : 5 ($\times 7$) $\rightarrow$ P : Q = 14 : 35
    \item R : P = 4 : 7 ($\times 2$) $\rightarrow$ R : P = 8 : 14
\end{itemize}
Unit P disamakan menjadi 14.

\begin{center}
\begin{tikzpicture}[scale=0.35, every node/.style={font=\footnotesize}]
    % R - Naikkan posisi y agar renggang
    \node[anchor=east] at (-1, 8) {Kota R (8 u)};
    \foreach \x in {0,...,7} \draw[fill=bargreen] (\x, 7) rectangle (\x+1, 9);
    % Label R
    \node[right] at (8.5, 8) {8 unit};
    
    % P
    \node[anchor=east] at (-1, 4) {Kota P (14 u)};
    \foreach \x in {0,...,13} \draw[fill=barblue] (\x, 3) rectangle (\x+1, 5);
    
    % Q
    \node[anchor=east] at (-1, 0) {Kota Q (35 u)};
    \foreach \x in {0,...,34} \draw[fill=barorange] (\x, -1) rectangle (\x+1, 1);
    
    % Comparison Brace between R and Q
    % Gunakan garis bantu putus-putus dari R ke bawah
    \draw[dashed, red, thick] (8, 7) -- (8, -1);
    
    % Brace selisih di bawah bar Q
    % Dari x=8 sampai x=35
    \draw[decorate,decoration={brace,amplitude=10pt,mirror},thick, red] (8, -1.5) -- (35, -1.5) node[midway, below=0.3cm, align=center] {Selisih = $35-8=27$ unit\\(8100 penduduk)};
\end{tikzpicture}
\end{center}

Dari model batang, terlihat selisih antara Kota Q dan Kota R:
\[ 35 \text{ unit} - 8 \text{ unit} = 27 \text{ unit} \]

Diketahui selisihnya adalah 8100 penduduk:
\begin{align*}
    27 \text{ unit} &= 8100 \\
    1 \text{ unit} &= \frac{8100}{27} = 300 \text{ penduduk}
\end{align*}

\textbf{Menghitung Kota R:}
Kota R memiliki 8 unit.
\[ R = 8 \times 300 = \mathbf{2.400} \text{ penduduk} \]

\newpage

\section*{Soal 3: Kecepatan (Speed)}
\textit{Sumber: Halaman 64, Soal 1}

\textbf{Soal:}
Nathaniel dan Trevor joging dari Taman E ke Taman F. Nathaniel butuh 2 jam untuk seluruh perjalanan. Trevor butuh 1 jam untuk menempuh $\frac{3}{5}$ perjalanan. Tentukan rasio kecepatan rata-rata Nathaniel terhadap Trevor.

\hrulefill

\textbf{Pembahasan dengan Bar Model:}

Kita bandingkan \textbf{jarak yang ditempuh dalam waktu yang sama} (1 jam).

\begin{center}
\begin{tikzpicture}[xscale=7, yscale=1.3, every node/.style={font=\small}]
    % Header Jarak Total
    \draw[thick, <->] (0, 3.2) -- (1, 3.2) node[midway, fill=white] {Jarak Total (Taman E ke F)};
    \draw[dashed, gray] (0, 3.2) -- (0, -1.5); % Garis start
    \draw[dashed, gray] (1, 3.2) -- (1, -1.5); % Garis finish
    
    % Nathaniel
    \node[anchor=south west] at (0, 2.1) {Nathaniel (Total = 2 jam):};
    % Draw 2 blocks of 1 hour
    \draw[fill=barblue] (0, 1.2) rectangle (0.5, 2) node[midway, white] {1 jam};
    \draw[fill=barblue] (0.5, 1.2) rectangle (1, 2) node[midway, white] {1 jam};
    % Keterangan di bawah bar Nathaniel
    \node[below] at (0.25, 1.2) {Jarak 1 Jam = $\frac{1}{2}$ Total};

    % Trevor
    \node[anchor=south west] at (0, -0.4) {Trevor (Dalam 1 jam):};
    % Trevor covers 3/5 in 1 hour
    \draw[fill=barorange] (0, -1.2) rectangle (0.6, -0.4) node[midway, white] {1 jam};
    \draw[fill=bargray] (0.6, -1.2) rectangle (1, -0.4) node[midway, black] {Sisa};
    % Keterangan di bawah bar Trevor
    \node[below] at (0.3, -1.2) {Jarak 1 Jam = $\frac{3}{5}$ Total};
    
    % Garis bantu perbandingan 1/2 vs 3/5
    \draw[dotted, thick, red] (0.5, 1.2) -- (0.5, -0.4);
    \draw[dotted, thick, red] (0.6, -0.4) -- (0.6, -1.5);
\end{tikzpicture}
\end{center}

\textbf{Analisis Model:}
\begin{itemize}
    \item Dalam waktu 1 jam, Nathaniel menempuh $\frac{1}{2}$ jarak total.
    \item Dalam waktu 1 jam, Trevor menempuh $\frac{3}{5}$ jarak total.
\end{itemize}

Rasio Kecepatan = Rasio Jarak (dalam waktu yang sama).
\[ \text{Kecepatan N} : \text{Kecepatan T} = \frac{1}{2} : \frac{3}{5} \]

Kalikan kedua sisi dengan 10 (KPK dari 2 dan 5):
\[ \left(\frac{1}{2} \times 10\right) : \left(\frac{3}{5} \times 10\right) = 5 : 6 \]

Jadi, rasio kecepatan rata-rata Nathaniel terhadap Trevor adalah \textbf{5 : 6}.

\newpage

\section*{Soal 4: Lingkaran (Area)}
\textit{Sumber: Halaman 75, Worked Example 1}

\textbf{Soal:}
Sebuah persegi dengan sisi 28 inci memiliki empat setengah lingkaran yang digambar pada sisi-sisinya, membentuk pola "daun" berempat di tengah. Berapakah luas daerah yang diarsir (daun)? ($\pi = \frac{22}{7}$)

\hrulefill

\textbf{Pembahasan dengan Model Komparasi:}

\begin{center}
\begin{tikzpicture}[scale=0.8, every node/.style={font=\small}]
    % Bar 1: 4 Semicircles
    \node[anchor=east, align=right] at (-0.2, 2) {Luas 4 Setengah\\Lingkaran:};
    \draw[fill=bargreen] (0, 1.5) rectangle (10, 2.5);
    \node[white, font=\bfseries] at (5, 2) {Area Total (Tumpang Tindih)};
    
    % Bar 2: Breakdown
    \node[anchor=east, align=right] at (-0.2, 0) {Komposisi:};
    \draw[fill=barblue] (0, -0.5) rectangle (6.5, 0.5) node[midway, white, font=\bfseries] {Luas Persegi};
    \draw[fill=barred] (6.5, -0.5) rectangle (10, 0.5) node[midway, white, font=\bfseries, align=center] {Luas Daun\\(Arsir)};
    
    % Braces/Lines showing equality
    \draw[dashed] (0, 1.5) -- (0, -0.5);
    \draw[dashed] (10, 1.5) -- (10, -0.5);
    \draw[dashed] (6.5, 1.5) -- (6.5, 0.5);
    
    % Brace keterangan
    \draw[decorate,decoration={brace,amplitude=5pt},thick] (0, 2.6) -- (10, 2.6) node[midway, above=0.2cm] {Luas 4 $\times$ $\frac{1}{2}$ Lingkaran = 2 Lingkaran Penuh};
\end{tikzpicture}
\end{center}

Logika visualnya: Jika kita menumpuk 4 setengah lingkaran, area tersebut menutupi seluruh persegi \textbf{ditambah} area daun (bagian yang tumpang tindih).

\[ \text{Luas Arsir} = \text{Luas 4 Setengah Lingkaran} - \text{Luas Persegi} \]

1.  \textbf{Luas Persegi ($s=28$):}
    $28 \times 28 = 784 \text{ in}^2$.

2.  \textbf{Luas 4 Setengah Lingkaran (Diameter 28 $\rightarrow$ Jari-jari 14):}
    $2 \times (\pi \times 14^2) = 2 \times \frac{22}{7} \times 196 = 1232 \text{ in}^2$.

3.  \textbf{Selisih:}
    $1232 - 784 = \mathbf{448} \text{ in}^2$.

\newpage

\section*{Soal 5: Volume}
\textit{Sumber: Halaman 92, Soal 9}

\textbf{Soal:}
Balok logam ($V = 196 cm^3$) di dalam tangki ($12 \times 15 \times 18$ cm). Keran mengisi air 11 menit hingga menutupi logam. Total isi (air + logam) saat itu $\frac{2}{5}$ kapasitas. Berapa lama lagi waktu untuk mengisi sisanya?

\hrulefill

\textbf{Pembahasan dengan Bar Model:}

\begin{center}
\begin{tikzpicture}[scale=1, every node/.style={font=\small}]
    % Tank Capacity Outline
    \node[anchor=south west] at (0, 4.2) {\textbf{Kapasitas Tangki (Dibagi 5 Bagian):}};
    \draw[thick] (0, 2) rectangle (10, 4);
    
    % Grid vertikal
    \foreach \x in {2,4,6,8} \draw[gray] (\x, 2) -- (\x, 4);
    
    % Filled Part (2/5)
    \draw[fill=barblue!50] (0, 2) rectangle (4, 4);
    \node[font=\bfseries] at (2, 3.2) {Terisi ($\frac{2}{5}$)};
    \node at (2, 2.6) {(Air + Logam)};
    
    % Empty Part (3/5)
    \node[font=\bfseries] at (7, 3.2) {Kosong ($\frac{3}{5}$)};
    \node at (7, 2.6) {(Harus diisi Air)};
    
    % Details visual with Arrows
    % Kiri
    \draw[->, thick, >=latex] (2, 1.8) -- (2, 1.2);
    \node[below, align=center, fill=gray!10, rounded corners, inner sep=5pt] at (2, 1.2) {
        \textbf{Volume 2 Bagian}\\
        $V_{\text{air}} + 196$\\
        Waktu keran = 11 menit\\
        (Hanya mengisi airnya)
    };
    
    % Kanan
    \draw[->, thick, >=latex] (7, 1.8) -- (7, 1.2);
    \node[below, align=center, fill=gray!10, rounded corners, inner sep=5pt] at (7, 1.2) {
        \textbf{Volume 3 Bagian}\\
        $V_{\text{kosong}}$ (Hanya Air)\\
        Waktu = \textbf{?}
    };
\end{tikzpicture}
\end{center}

1.  \textbf{Hitung Volume Total Tangki:}
    $12 \times 15 \times 18 = 3240 \text{ cm}^3$.

2.  \textbf{Analisis Bagian Terisi (2 Unit):}
    Volume 2 unit = $\frac{2}{5} \times 3240 = 1296 \text{ cm}^3$.
    Volume Air yang masuk = $1296 - 196 \text{ (logam)} = 1100 \text{ cm}^3$.
    
    \textbf{Debit Keran:} Keran mengisi 1100 cm$^3$ dalam 11 menit.
    \[ \text{Debit} = \frac{1100}{11} = 100 \text{ cm}^3/\text{menit} \]

3.  \textbf{Analisis Bagian Kosong (3 Unit):}
    Kita perlu mengisi 3 unit sisanya \textbf{murni dengan air}.
    Volume bagian kosong = Total - Terisi = $3240 - 1296 = 1944 \text{ cm}^3$.

4.  \textbf{Waktu Tambahan:}
    \[ \text{Waktu} = \frac{\text{Volume Kosong}}{\text{Debit}} = \frac{1944}{100} = 19,44 \text{ menit} \]
    
Jadi, waktu yang dibutuhkan adalah \textbf{19,44 menit} (atau 19 menit 26,4 detik).

\end{document}
