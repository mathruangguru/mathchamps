\documentclass[a4paper, 12pt]{article}
\usepackage[margin=2.5cm]{geometry}
\usepackage{amsmath}
\usepackage{amssymb}
\usepackage{graphicx}
\usepackage{enumitem}
\usepackage[indonesian]{babel}
\usepackage{parskip}
\usepackage{tikz}
\usetikzlibrary{decorations.pathreplacing, calc, positioning, arrows.meta, patterns}

% Definisi warna pastel
\definecolor{colP}{RGB}{100, 149, 237}   % Biru
\definecolor{colQ}{RGB}{255, 179, 71}    % Oranye
\definecolor{colR}{RGB}{60, 179, 113}    % Hijau
\definecolor{colGray}{RGB}{220, 220, 220} % Abu-abu

% Judul Dokumen
\title{\textbf{Pembahasan Soal Matematika Kelas 6 SD}\\ \textit{(Soal Cerita)}}
\author{by K-12 Subject Excellence, Ruangguru}
\date{}

\begin{document}

\maketitle

\section*{Soal 1: Aljabar (Algebra)}

\textbf{Soal:}
Terdapat $x$ permen di dalam sebuah tas. Jane mengambil beberapa permen dari tas tersebut. Ruth mengambil dua kali lipat jumlah permen yang diambil Jane. Sally mengambil 25 permen lebih banyak daripada Jane. Tas tersebut sekarang kosong.
\begin{enumerate}[label=(\alph*)]
    \item Nyatakan jumlah permen yang diambil Jane dalam bentuk $x$.
    \item Jika $x = 265$, tentukan jumlah permen yang diambil oleh masing-masing anak.
\end{enumerate}

\hrulefill

\textbf{Pembahasan:}

Kita gambarkan setiap unit sebagai satu kotak utuh.

\begin{center}
\begin{tikzpicture}[scale=0.8, every node/.style={font=\small}]
    % Jane: 1 kotak
    \node[anchor=east, font=\bfseries] at (0, 3) {Jane:};
    \draw[fill=colP, draw=white, line width=1pt] (0, 2.5) rectangle (2, 3.5);
    \draw[thick] (0, 2.5) rectangle (2, 3.5); % Border luar
    \node[white, font=\bfseries] at (1, 3) {1};
    
    % Ruth: 2 kotak
    \node[anchor=east, font=\bfseries] at (0, 1) {Ruth:};
    \foreach \x in {0, 2} {
        \draw[fill=colP, draw=white, line width=1pt] (\x, 0.5) rectangle (\x+2, 1.5);
    }
    \draw[thick] (0, 0.5) rectangle (4, 1.5); % Border luar
    \node[white, font=\bfseries] at (1, 1) {1};
    \node[white, font=\bfseries] at (3, 1) {1};
    
    % Sally: 1 kotak + 25
    \node[anchor=east, font=\bfseries] at (0, -1.5) {Sally:};
    \draw[fill=colP, draw=white, line width=1pt] (0, -2) rectangle (2, -1);
    \draw[thick] (0, -2) rectangle (2, -1);
    \node[white, font=\bfseries] at (1, -1.5) {1};
    
    % Bagian +25 (beda warna/ukuran)
    \draw[fill=colQ, draw=black, thick] (2.2, -2) rectangle (3.7, -1) node[midway, white, font=\bfseries] {25};
    
    % Total Brace
    \draw[decorate,decoration={brace,amplitude=10pt,mirror},thick] (4.5, -2.2) -- (4.5, 3.7) node[midway, right=0.5cm, align=left] {\textbf{Total} $= x$};
\end{tikzpicture}
\end{center}

Terlihat total unit kotak biru ada 4 buah.
\[ x = 4 \text{ kotak} + 25 \]

\textbf{(a) Menyatakan Jane dalam $x$}
Karena Jane memiliki tepat 1 kotak (1 unit):
\begin{align*}
    4 \text{ unit} + 25 &= x \\
    4 \text{ unit} &= x - 25 \\
    1 \text{ unit} &= \frac{x - 25}{4}
\end{align*}
Jadi, permen Jane = $\boxed{\frac{x - 25}{4}}$.

\textbf{(b) Jika $x = 265$}
\begin{align*}
    4 \text{ unit} &= 265 - 25 \\
    4 \text{ unit} &= 240 \\
    1 \text{ unit} &= 60
\end{align*}

Maka:
\begin{itemize}
    \item \textbf{Jane} (1 unit) = \textbf{60}
    \item \textbf{Ruth} (2 unit) = $60 \times 2$ = \textbf{120}
    \item \textbf{Sally} (1 unit + 25) = $60 + 25$ = \textbf{85}
\end{itemize}

\newpage

\section*{Soal 2: Perbandingan (Ratio)}

\textbf{Soal:}
Kota P memiliki $\frac{2}{5}$ jumlah penduduk Kota Q. Perbandingan jumlah penduduk di Kota R terhadap Kota P adalah $4:7$. Kota Q memiliki 8100 penduduk lebih banyak daripada Kota R. Berapa jumlah penduduk di Kota R?

\hrulefill

\textbf{Pembahasan:}

Kita perlu menyamakan "ukuran kotak" (unit) untuk Kota P agar bisa membandingkan ketiga kota.
\begin{itemize}
    \item P : Q = 2 : 5 (Kalikan 7) $\rightarrow$ P : Q = 14 : 35
    \item R : P = 4 : 7 (Kalikan 2) $\rightarrow$ R : P = 8 : 14
\end{itemize}
Sekarang unit P sudah sama (14 unit).

\begin{center}
\begin{tikzpicture}[scale=0.35, every node/.style={font=\footnotesize}]
    % R (8 unit)
    \node[anchor=east] at (-0.5, 7) {\textbf{Kota R} (8 unit)};
    \draw[fill=colR] (0, 6) rectangle (8, 8);
    \foreach \x in {1,...,7} \draw[white, line width=0.8pt] (\x, 6) -- (\x, 8); % Grid lines
    \draw[thick] (0, 6) rectangle (8, 8); % Border
    
    % P (14 unit)
    \node[anchor=east] at (-0.5, 3.5) {\textbf{Kota P} (14 unit)};
    \draw[fill=colP] (0, 2.5) rectangle (14, 4.5);
    \foreach \x in {1,...,13} \draw[white, line width=0.8pt] (\x, 2.5) -- (\x, 4.5); % Grid lines
    \draw[thick] (0, 2.5) rectangle (14, 4.5); % Border
    
    % Q (35 unit)
    \node[anchor=east] at (-0.5, 0) {\textbf{Kota Q} (35 unit)};
    \draw[fill=colQ] (0, -1) rectangle (35, 1);
    \foreach \x in {1,...,34} \draw[white, line width=0.8pt] (\x, -1) -- (\x, 1); % Grid lines
    \draw[thick] (0, -1) rectangle (35, 1); % Border
    
    % Garis putus-putus penanda selisih
    \draw[dashed, red, thick] (8, 6) -- (8, -1.5);
    \draw[dashed, red, thick] (35, -1) -- (35, -1.5);
    
    % Brace Selisih
    \draw[decorate,decoration={brace,amplitude=10pt,mirror},thick, red] (8, -2) -- (35, -2) 
    node[midway, below=0.5cm, align=center, font=\bfseries, color=red] {Selisih = 27 kotak\\(8100 orang)};
\end{tikzpicture}
\end{center}

Dari gambar grid di atas:
Selisih panjang batang Q dan R adalah $35 - 8 = 27$ kotak.

\[ 27 \text{ unit} = 8100 \implies 1 \text{ unit} = 300 \]

\textbf{Kota R (8 unit):}
\[ 8 \times 300 = \mathbf{2.400} \text{ penduduk} \]

\newpage

\section*{Soal 3: Kecepatan (Speed)}

\textbf{Soal:}
Nathaniel dan Trevor masing-masing lari santai (joging) dari Taman E ke Taman F. Nathaniel membutuhkan waktu 2 jam untuk menempuh seluruh perjalanan, sedangkan Trevor membutuhkan waktu 1 jam untuk menempuh tiga perlima perjalanan tersebut. Tentukan rasio kecepatan rata-rata Nathaniel terhadap kecepatan rata-rata Trevor.

\hrulefill

\textbf{Pembahasan:}

Kita bandingkan \textbf{jarak yang ditempuh dalam waktu yang sama}, yaitu 1 jam.
Kita anggap Jarak Total = 10 unit jarak (KPK dari 2 dan 5 untuk mempermudah visualisasi).

\begin{center}
\begin{tikzpicture}[scale=0.8, every node/.style={font=\small}]
    % --- Header Jarak Total ---
    \node[anchor=south west, font=\bfseries] at (0, 6.2) {Jarak Total (dimisalkan 10 km/unit):};
    % Kotak Jarak Total (Kosong/Grid)
    \draw[thick] (0, 5.5) rectangle (10, 6);
    \foreach \x in {1,...,9} \draw[gray] (\x, 5.5) -- (\x, 6);
    
    % --- Nathaniel ---
    \node[anchor=south west] at (0, 4.6) {Nathaniel (1 Jam menempuh $\frac{1}{2}$):};
    % Batang Nathaniel (5 kotak)
    \draw[fill=colP] (0, 3.5) rectangle (5, 4.5);
    \foreach \x in {1,...,4} \draw[white, line width=0.8pt] (\x, 3.5) -- (\x, 4.5);
    \draw[thick] (0, 3.5) rectangle (5, 4.5); % Border
    % Label Jumlah Kotak
    \node[right, colP, font=\bfseries] at (5.2, 4) {5 kotak};

    % --- Trevor ---
    \node[anchor=south west] at (0, 2.1) {Trevor (1 Jam menempuh $\frac{3}{5}$):};
    % Batang Trevor (6 kotak)
    \draw[fill=colQ] (0, 1) rectangle (6, 2);
    \foreach \x in {1,...,5} \draw[white, line width=0.8pt] (\x, 1) -- (\x, 2);
    \draw[thick] (0, 1) rectangle (6, 2); % Border
    % Label Jumlah Kotak
    \node[right, colQ, font=\bfseries] at (6.2, 1.5) {6 kotak};
    
    % --- Garis Bantu ---
    \draw[dashed, gray] (5, 4.5) -- (5, 0.5); % Garis batas Nathaniel
    \draw[dashed, gray] (6, 2) -- (6, 0.5);   % Garis batas Trevor
\end{tikzpicture}
\end{center}

Rasio Kecepatan = Rasio Jarak (dalam 1 jam)
\[ \text{Nathaniel} : \text{Trevor} = 5 \text{ kotak} : 6 \text{ kotak} = \mathbf{5 : 6} \]

\newpage

\section*{Soal 4: Rasio Bertingkat (Ratio)}

\textbf{Soal:}
Rasio jumlah buku Claude terhadap buku Robin adalah $5:6$. Rasio jumlah buku Robin terhadap buku Ian adalah $7:4$. Jika Claude memiliki 22 buku lebih banyak daripada Ian, berapa banyak buku yang dimiliki Robin?

\hrulefill

\textbf{Pembahasan:}

Variabel penghubung adalah **Robin**. Kita harus menyamakan unit Robin pada kedua rasio.
\begin{itemize}
    \item C : R = 5 : 6 (Kalikan 7) $\rightarrow$ \textbf{35 : 42}
    \item R : I = 7 : 4 (Kalikan 6) $\rightarrow$ \textbf{42 : 24}
\end{itemize}
Rasio gabungan C : R : I = 35 : 42 : 24.

\begin{center}
\begin{tikzpicture}[scale=0.25, every node/.style={font=\footnotesize}]
    % Claude (35 unit)
    \node[anchor=east] at (-0.5, 8) {\textbf{Claude} (35 u)};
    \draw[fill=colP] (0, 7) rectangle (35, 9);
    \foreach \x in {1,...,34} \draw[white, line width=0.5pt] (\x, 7) -- (\x, 9); % Grid halus
    \draw[thick] (0, 7) rectangle (35, 9);
    
    % Robin (42 unit)
    \node[anchor=east] at (-0.5, 4) {\textbf{Robin} (42 u)};
    \draw[fill=colQ] (0, 3) rectangle (42, 5);
    \foreach \x in {1,...,41} \draw[white, line width=0.5pt] (\x, 3) -- (\x, 5);
    \draw[thick] (0, 3) rectangle (42, 5);
    
    % Ian (24 unit)
    \node[anchor=east] at (-0.5, 0) {\textbf{Ian} (24 u)};
    \draw[fill=colR] (0, -1) rectangle (24, 1);
    \foreach \x in {1,...,23} \draw[white, line width=0.5pt] (\x, -1) -- (\x, 1);
    \draw[thick] (0, -1) rectangle (24, 1);
    
    % Penanda Selisih
    \draw[dashed, red, thick] (24, 7) -- (24, -1.5);
    \draw[dashed, red, thick] (35, 7) -- (35, -1.5);
    
    \draw[decorate,decoration={brace,amplitude=8pt},thick, red] (24, 9.5) -- (35, 9.5) 
    node[midway, above=0.5cm, align=center, font=\bfseries, color=red] {Selisih = 11 unit\\(22 buku)};
\end{tikzpicture}
\end{center}

Selisih unit Claude dan Ian: $35 - 24 = 11$ unit.
\[ 11 \text{ unit} = 22 \text{ buku} \implies 1 \text{ unit} = 2 \text{ buku} \]

\textbf{Jumlah buku Robin (42 unit):}
\[ 42 \times 2 = \mathbf{84} \text{ buku} \]

\newpage

\section*{Soal 5: Pecahan Sisa (Fraction of Remainder)}

\textbf{Soal:}
Gavin memiliki sejumlah uang. Ia menyumbangkan $\frac{1}{2}$ dari uangnya ke Yayasan A. Kemudian, ia menyumbangkan $\frac{1}{4}$ dari sisanya ke Palang Merah. Selanjutnya, ia memberikan $\frac{1}{3}$ dari sisanya lagi ke Mercy Corps. Terakhir, ia mendonasikan $\frac{1}{2}$ dari sisanya ke Global Fund. Gavin tersisa \$625 pada akhirnya. Berapa uang Gavin mula-mula?

\hrulefill

\textbf{Pembahasan (Working Backwards):}

Kita gambar kotaknya secara harfiah untuk melihat sisa dengan metode mundur.

\begin{center}
\begin{tikzpicture}[scale=0.9, every node/.style={font=\small}]
    % --- Bar 4: Sisa 3 ---
    \node[anchor=west, font=\bfseries, scale=1.1] at (0, 1.5) {4. Tahap Terakhir (Sisa 3)};
    % Sisa akhir 1/2 bagian. Kita gambar Sisa 3 sebagai 2 kotak.
    \draw[fill=colR] (0, 0) rectangle (2, 1);
    \draw[white, line width=1.5pt] (1, 0) -- (1, 1);
    \draw[thick] (0, 0) rectangle (2, 1);
    
    \node at (0.5, 0.5) {\scriptsize Donasi};
    \node[font=\bfseries] at (1.5, 0.5) {\$625};
    
    % Brace Sisa Akhir
    \draw[decorate,decoration={brace,amplitude=6pt,mirror},thick] (1, -0.2) -- (2, -0.2);
    
    % Penjelasan Kanan
    \node[right, align=left] at (2.5, 0.5) {
        $\leftarrow$ Diketahui sisa akhir = \$625.\\
        Ini adalah 1 unit.\\
        \textbf{Total Sisa 3 = \$625 $\times$ 2 = \$1250}
    };

    % Panah penghubung ke atas
    \draw[dashed, ->, thick, gray] (1, 1.2) -- (2, 3.3);

    % --- Bar 3: Sisa 2 ---
    \node[anchor=west, font=\bfseries, scale=1.1] at (0, 5) {3. Tahap Ketiga (Sisa 2)};
    % Sisa 3 mewakili 2/3 bagian (karena 1/3 didonasikan).
    % Kita gambar Sisa 2 sebagai 3 kotak.
    \draw[fill=colQ] (0, 3.5) rectangle (3, 4.5);
    \foreach \x in {1,2} \draw[white, line width=1.5pt] (\x, 3.5) -- (\x, 4.5);
    \draw[thick] (0, 3.5) rectangle (3, 4.5);
    
    \node at (0.5, 4) {\scriptsize Donasi};
    \node[font=\bfseries] at (2, 4) {Sisa 3};
    
    % Brace Sisa 3 (Mencakup 2 kotak kanan)
    \draw[decorate,decoration={brace,amplitude=6pt,mirror},thick] (1, 3.3) -- (3, 3.3);
    
    % Penjelasan Kanan
    \node[right, align=left] at (3.5, 4) {
        $\leftarrow$ Sisa 3 (\$1250) adalah 2 unit.\\
        1 unit = \$625.\\
        \textbf{Total Sisa 2 = \$625 $\times$ 3 = \$1875}
    };

    % Panah penghubung ke atas
    \draw[dashed, ->, thick, gray] (2, 4.7) -- (2.5, 6.8);

    % --- Bar 2: Sisa 1 ---
    \node[anchor=west, font=\bfseries, scale=1.1] at (0, 8.5) {2. Tahap Kedua (Sisa 1)};
    % Sisa 2 mewakili 3/4 bagian (karena 1/4 didonasikan).
    % Kita gambar Sisa 1 sebagai 4 kotak.
    \draw[fill=colP] (0, 7) rectangle (4, 8);
    \foreach \x in {1,2,3} \draw[white, line width=1.5pt] (\x, 7) -- (\x, 8);
    \draw[thick] (0, 7) rectangle (4, 8);
    
    \node at (0.5, 7.5) {\scriptsize Donasi};
    \node[font=\bfseries] at (2.5, 7.5) {Sisa 2};
    
    % Brace Sisa 2 (Mencakup 3 kotak kanan)
    \draw[decorate,decoration={brace,amplitude=6pt,mirror},thick] (1, 6.8) -- (4, 6.8);
    
    % Penjelasan Kanan
    \node[right, align=left] at (4.5, 7.5) {
        $\leftarrow$ Sisa 2 (\$1875) adalah 3 unit.\\
        1 unit = \$625.\\
        \textbf{Total Sisa 1 = \$625 $\times$ 4 = \$2500}
    };

    % Panah penghubung ke atas
    \draw[dashed, ->, thick, gray] (3, 8.2) -- (2, 10.3);

    % --- Bar 1: Mula-mula ---
    \node[anchor=west, font=\bfseries, scale=1.1] at (0, 12) {1. Uang Mula-mula};
    % Sisa 1 mewakili 1/2 bagian (karena 1/2 didonasikan). 
    % Artinya Mula-mula = 2 kotak.
    \draw[fill=colGray] (0, 10.5) rectangle (4, 11.5);
    \draw[white, line width=1.5pt] (2, 10.5) -- (2, 11.5);
    \draw[thick] (0, 10.5) rectangle (4, 11.5);
    
    \node at (1, 11) {\scriptsize Donasi};
    \node[font=\bfseries] at (3, 11) {Sisa 1};
    
    % Brace Sisa 1
    \draw[decorate,decoration={brace,amplitude=6pt,mirror},thick] (2, 10.3) -- (4, 10.3);
    
    % Penjelasan Kanan
    \node[right, align=left] at (4.5, 11) {
        $\leftarrow$ Sisa 1 mewakili 1 kotak.\\
        Maka, \textbf{Total = \$2500 $\times$ 2 = \$5000}
    };

\end{tikzpicture}
\end{center}

Jadi, uang Gavin mula-mula adalah \textbf{\$5.000}.

\end{document}
