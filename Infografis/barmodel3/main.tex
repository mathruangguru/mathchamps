\documentclass[a4paper, 12pt]{article}
\usepackage[margin=2.5cm]{geometry}
\usepackage{amsmath}
\usepackage{amssymb}
\usepackage{graphicx}
\usepackage{enumitem}
\usepackage[indonesian]{babel}
\usepackage{parskip}
\usepackage{tikz}
\usetikzlibrary{decorations.pathreplacing, calc, positioning, arrows.meta, patterns, shapes.geometric}

% --- PALET WARNA PASTEL (Ramah Anak) ---
\definecolor{colBlue}{RGB}{100, 149, 237}    % Biru Unit Utama
\definecolor{colOrange}{RGB}{255, 179, 71}   % Oranye Unit Pembanding
\definecolor{colGreen}{RGB}{60, 179, 113}    % Hijau untuk Penanda/Sisa
\definecolor{colGray}{RGB}{220, 220, 220}    % Abu-abu untuk Unit Bayangan/Hilang
\definecolor{colRed}{RGB}{255, 99, 71}       % Merah untuk Pengurangan/Selisih

\title{\textbf{Pembahasan Soal Matematika Kelas 5}\\ \large (Visualisasi Bar Model Unit Persegi)}
\author{}
\date{}

\begin{document}

\maketitle

% ==========================================================================================
% SOAL 1
% ==========================================================================================
\section*{Soal 1: Bilangan Cacah (Quotient \& Remainder)}
\textbf{Soal:}
Ketika Bilangan X dibagi dengan Bilangan Y, hasil baginya adalah 16 dan sisanya 3. Jumlah dari kedua bilangan, hasil bagi, dan sisanya adalah 345. Berapakah Bilangan X?

\hrulefill

\textbf{Pembahasan:}

Dari soal, kita tahu hubungan: $X = (16 \times Y) + 3$.
Jika kita gambarkan dalam kotak unit:
\begin{itemize}
    \item \textbf{Bilangan Y} = 1 kotak unit.
    \item \textbf{Bilangan X} = 16 kotak unit + nilai 3.
\end{itemize}

Persamaan totalnya:
\[ X + Y + 16 + 3 = 345 \]
\[ (16 \text{ unit} + 3) + (1 \text{ unit}) + 19 = 345 \]
\[ 17 \text{ unit} + 22 = 345 \]
\[ 17 \text{ unit} = 323 \]

\vspace{0.5cm}
\begin{center}
\begin{tikzpicture}[scale=0.6, every node/.style={font=\small}]
    % --- Bar Y ---
    \node[anchor=east] at (0, 3) {\textbf{Y} (1 unit)};
    \draw[fill=colOrange, draw=white, line width=1pt] (0, 2.5) rectangle (1, 3.5);
    \draw[thick] (0, 2.5) rectangle (1, 3.5); % Border
    \node at (0.5, 3) {};

    % --- Bar X ---
    \node[anchor=east] at (0, 1) {\textbf{X} (16 unit + 3)};
    % Loop menggambar 16 kotak biru
    \foreach \x in {0, ..., 15} {
        \draw[fill=colBlue, draw=white, line width=1pt] (\x, 0.5) rectangle (\x+1, 1.5);
    }
    \draw[thick] (0, 0.5) rectangle (16, 1.5); % Border luar unit
    
    % Kotak sisa 3 (Beda warna & ukuran fix visual)
    \draw[fill=colGreen, draw=black, thick] (16.2, 0.5) rectangle (17.2, 1.5);
    \node[white, font=\bfseries] at (16.7, 1) {3};

    % Label Unit
    \node[white] at (0.5, 1) {};
    \node[white] at (15.5, 1) {};
    \node at (8, 1) {};

    % Brace Total
    \draw[decorate,decoration={brace,amplitude=10pt,mirror},thick] (17.5, 0.5) -- (17.5, 3.5) 
    node[midway, right=0.6cm, align=left, text width=4cm] {Total Nilai X + Y\\$= 345 - 19$\\$= \mathbf{323}$};
\end{tikzpicture}
\end{center}
\vspace{0.5cm}

\textbf{Hitungan:}
\[ 1 \text{ unit} = 323 \div 17 = 19 \quad (\text{Nilai Y}) \]
\[ X = (16 \times 19) + 3 = 304 + 3 = \mathbf{307} \]

\newpage

% ==========================================================================================
% SOAL 2 (PERBAIKAN: TOTAL KE KANAN)
% ==========================================================================================
\section*{Soal 2: Pecahan (Equal Fractions)}
\textbf{Soal:}
Di Peternakan A, $\frac{4}{5}$ dari jumlah domba sama dengan $\frac{1}{2}$ dari jumlah domba di Peternakan B. Total domba di Peternakan A dan Peternakan B adalah 845 ekor. Berapa jumlah domba di Peternakan B?

\hrulefill

\textbf{Pembahasan:}

Samakan pembilang pecahan agar unit pembandingnya setara (Equal Units).
\begin{itemize}
    \item Peternakan A: $\frac{4}{5}$ (4 kotak dari total 5).
    \item Peternakan B: $\frac{1}{2} = \frac{4}{8}$ (4 kotak dari total 8).
\end{itemize}
Jadi, kita gambar 4 kotak yang "sama besar" untuk kedua peternakan sebagai jembatan.

\vspace{0.5cm}
\begin{center}
\begin{tikzpicture}[scale=0.7, every node/.style={font=\small}]
    % --- Peternakan A ---
    \node[anchor=east] at (0, 4) {\textbf{Peternakan A}};
    % 4 Unit Setara (Biru)
    \foreach \x in {0, ..., 3} {
        \draw[fill=colBlue, draw=white, line width=1pt] (\x, 3.5) rectangle (\x+1, 4.5);
    }
    % 1 Unit Sisa (Biru Muda - bagian dari total A)
    \draw[fill=colBlue!40, draw=white, line width=1pt] (4, 3.5) rectangle (5, 4.5);
    \draw[thick] (0, 3.5) rectangle (5, 4.5); % Border
    \node[right] at (5.2, 4) {Total = 5 unit};

    % --- Peternakan B ---
    \node[anchor=east] at (0, 1.5) {\textbf{Peternakan B}};
    % 4 Unit Setara (Biru) - Sejajar dengan atas
    \foreach \x in {0, ..., 3} {
        \draw[fill=colBlue, draw=white, line width=1pt] (\x, 1) rectangle (\x+1, 2);
    }
    % 4 Unit Sisa (Oranye - total B ada 8)
    \foreach \x in {4, ..., 7} {
        \draw[fill=colOrange, draw=white, line width=1pt] (\x, 1) rectangle (\x+1, 2);
    }
    \draw[thick] (0, 1) rectangle (8, 2); % Border
    \node[right] at (8.2, 1.5) {Total = 8 unit};

    % Brace Hubungan (Atas & Bawah)
    \draw[decorate,decoration={brace,amplitude=5pt,raise=3pt}, thick, color=colBlue] (0, 4.5) -- (4, 4.5) node[midway, above=0.4cm] {$\frac{4}{5}$ Bagian A};
    \draw[decorate,decoration={brace,amplitude=5pt,mirror,raise=3pt}, thick, color=colBlue] (0, 1) -- (4, 1) node[midway, below=0.4cm] {$\frac{4}{8}$ Bagian B (Sama Besar)};

    % Brace Total Keseluruhan (DI KANAN)
    % Menjangkau dari bawah B (y=1) sampai atas A (y=4.5)
    \draw[decorate,decoration={brace,amplitude=10pt,mirror}, thick] (12, 1) -- (12, 4.5) 
    node[midway, right=0.5cm, align=left] {\textbf{Total Domba}\\845 ekor};

\end{tikzpicture}
\end{center}
\vspace{0.5cm}

\textbf{Hitungan:}
Total seluruh kotak unit = 5 (A) + 8 (B) = 13 unit.
\[ 13 \text{ unit} = 845 \]
\[ 1 \text{ unit} = 845 \div 13 = 65 \text{ ekor} \]
Peternakan B memiliki 8 unit:
\[ 8 \times 65 = \mathbf{520} \text{ ekor} \]

\newpage

% ==========================================================================================
% SOAL 3
% ==========================================================================================
\section*{Soal 3: Rasio (Before and After)}
\textbf{Soal:}
Rasio uang Terry terhadap uang Maria awalnya adalah 4:9. Setelah Terry membelanjakan setengah uangnya dan Maria membelanjakan \$20, uang Maria menjadi dua kali lipat uang Terry. Berapa uang Terry mula-mula?

\hrulefill

\textbf{Pembahasan:}

Gunakan kotak unit untuk melacak perubahan.
\begin{itemize}
    \item \textbf{Awal:} Terry (4 kotak), Maria (9 kotak).
    \item \textbf{Aksi:} Terry sisa 2 kotak (karena setengahnya habis).
    \item \textbf{Syarat Akhir:} Uang Maria = 2 $\times$ Uang Terry. 
    Karena sisa Terry adalah 2 kotak, maka sisa Maria haruslah $2 \times 2 = \mathbf{4 \text{ kotak}}$.
\end{itemize}

\vspace{0.5cm}
\begin{center}
\begin{tikzpicture}[scale=0.7, every node/.style={font=\small}]
    % --- Terry Awal ---
    \node[anchor=east] at (0, 6) {Terry (Awal):};
    \foreach \x in {0, ..., 3} {
        \draw[fill=colBlue, draw=white, line width=1pt] (\x, 5.5) rectangle (\x+1, 6.5);
    }
    \draw[thick] (0, 5.5) rectangle (4, 6.5); 

    % --- Terry Akhir ---
    \node[anchor=east] at (0, 4) {Terry (Sisa):};
    % 2 Kotak Sisa
    \foreach \x in {0, 1} {
        \draw[fill=colBlue, draw=white, line width=1pt] (\x, 3.5) rectangle (\x+1, 4.5);
    }
    % 2 Kotak Hilang (Dashed)
    \draw[dashed, draw=gray] (2, 3.5) rectangle (4, 4.5);
    \node[gray] at (3, 4) {Habis};
    \draw[thick] (0, 3.5) rectangle (2, 4.5); % Border sisa

    % --- Maria Awal ---
    \node[anchor=east] at (0, 2) {Maria (Awal):};
    \foreach \x in {0, ..., 8} {
        \draw[fill=colOrange, draw=white, line width=1pt] (\x, 1.5) rectangle (\x+1, 2.5);
    }
    \draw[thick] (0, 1.5) rectangle (9, 2.5);

    % --- Maria Akhir ---
    \node[anchor=east] at (0, 0) {Maria (Sisa):};
    % 4 Kotak Sisa (Target)
    \foreach \x in {0, ..., 3} {
        \draw[fill=colOrange, draw=white, line width=1pt] (\x, -0.5) rectangle (\x+1, 0.5);
    }
    \draw[thick] (0, -0.5) rectangle (4, 0.5);
    
    % 5 Kotak Hilang ($20)
    \foreach \x in {4, ..., 8} {
        \draw[pattern=north east lines, pattern color=colRed, draw=colRed] (\x, -0.5) rectangle (\x+1, 0.5);
    }
    
    % Brace Penjelasan Logika
    \draw[decorate,decoration={brace,amplitude=5pt,mirror}, thick, red] (4, -0.7) -- (9, -0.7) 
    node[midway, below=0.3cm] {\textbf{5 Unit = \$20}};
    
    \node[right, align=left] at (9.2, 0) {$\leftarrow$ Sisa ini harus 4 unit\\agar 2x lipat sisa Terry};

\end{tikzpicture}
\end{center}
\vspace{0.5cm}

\textbf{Hitungan:}
Dari gambar Maria (Akhir), selisih kotak awal (9) dan kotak sisa (4) adalah 5 kotak.
5 kotak ini mewakili uang yang dibelanjakan (\$20).
\[ 5 \text{ kotak} = \$20 \implies 1 \text{ kotak} = \$4 \]
Uang Terry Mula-mula (4 kotak):
\[ 4 \times \$4 = \mathbf{\$16} \]

\newpage

% ==========================================================================================
% SOAL 4
% ==========================================================================================
\section*{Soal 4: Perbandingan Dua Skenario (Gap \& Difference)}
\textbf{Soal:}
Ryan dan Marie memiliki sejumlah kelereng. 
\begin{itemize}
    \item \textbf{Skenario 1:} Jika Ryan kehilangan 15 kelereng, rasionya Ryan:Marie = 4:1.
    \item \textbf{Skenario 2:} Jika Ryan kehilangan 75 kelereng, rasionya Ryan:Marie = 3:2.
\end{itemize}
Berapa banyak kelereng Ryan mula-mula?

\hrulefill

\textbf{Pembahasan:}

Jumlah kelereng Marie **TETAP**. Kita jadikan Marie sebagai patokan unit yang sama.
\begin{itemize}
    \item Skenario 1: Ryan : Marie = 4 : 1. (Ubah jadi \textbf{8 : 2} agar Marie sama dengan skenario 2).
    \item Skenario 2: Ryan : Marie = \textbf{3 : 2}.
\end{itemize}
Jadi, kita gunakan **2 kotak** untuk Marie di kedua gambar.

\vspace{0.5cm}
\begin{center}
\begin{tikzpicture}[scale=0.7, every node/.style={font=\small}]
    % --- Marie (Konstan) ---
    \node[anchor=east] at (0, 6) {\textbf{Marie} (2 unit)};
    \foreach \x in {0, 1} {
        \draw[fill=colGray, draw=white, line width=1pt] (\x, 5.5) rectangle (\x+1, 6.5);
    }
    \draw[thick] (0, 5.5) rectangle (2, 6.5);

    % --- Ryan Skenario 1 ---
    \node[anchor=east] at (0, 4) {Ryan (Kurang 15)};
    % 8 Unit
    \foreach \x in {0, ..., 7} {
        \draw[fill=colBlue, draw=white, line width=1pt] (\x, 3.5) rectangle (\x+1, 4.5);
    }
    \draw[thick] (0, 3.5) rectangle (8, 4.5);
    % Kotak putus-putus +15 (Posisi relatif di kanan)
    \draw[dashed] (8, 3.5) rectangle (9.2, 4.5);
    \node at (8.6, 4) {15};

    % --- Ryan Skenario 2 ---
    \node[anchor=east] at (0, 1.5) {Ryan (Kurang 75)};
    % 3 Unit
    \foreach \x in {0, ..., 2} {
        \draw[fill=colOrange, draw=white, line width=1pt] (\x, 1) rectangle (\x+1, 2);
    }
    \draw[thick] (0, 1) rectangle (3, 2);
    % Kotak putus-putus +75
    \draw[dashed] (3, 1) rectangle (9.2, 2);
    \node at (6.1, 1.5) {75 (Total yang hilang)};

    % --- Analisis GAP ---
    % Garis bantu vertikal
    \draw[dotted, thick, red] (3, 0.5) -- (3, 3.5);
    \draw[dotted, thick, red] (8, 0.5) -- (8, 3.5);
    
    % Brace Selisih Unit
    \draw[decorate,decoration={brace,amplitude=5pt}, thick, colRed] (3, 4.6) -- (8, 4.6) 
    node[midway, above=0.2cm] {\textbf{Selisih = 5 Unit}};
    
    % Brace Selisih Nilai
    \draw[decorate,decoration={brace,amplitude=5pt,mirror}, thick, colRed] (3, 0.8) -- (8, 0.8) 
    node[midway, below=0.2cm] {\textbf{Selisih Nilai = $75 - 15 = 60$}};

\end{tikzpicture}
\end{center}
\vspace{0.5cm}

\textbf{Hitungan:}
Dari gambar, terlihat perbedaan panjang kotak Ryan (5 unit) setara dengan perbedaan nilai kehilangan (60).
\[ 5 \text{ unit} = 60 \implies 1 \text{ unit} = 12 \]
Ryan Mula-mula (Lihat Skenario 1):
\[ (8 \text{ unit} \times 12) + 15 = 96 + 15 = \mathbf{111} \]

\newpage

% ==========================================================================================
% SOAL 5
% ==========================================================================================
\section*{Soal 5: Desimal \& Logika (Assumption Method)}
\textbf{Soal:}
Sebuah perusahaan transportasi mengirimkan 78 piring. Mereka dibayar \$1.50 untuk setiap piring yang terkirim utuh, namun harus membayar ganti rugi \$9.50 untuk setiap piring yang pecah. Jika perusahaan tersebut menerima total \$73, berapa banyak piring yang pecah?

\hrulefill

\textbf{Pembahasan:}

Kita gunakan \textit{Assumption Method} (Metode Pengandaian). Kita bandingkan kondisi "Sempurna" dengan kondisi "Nyata" menggunakan diagram batang.

\vspace{0.5cm}
\begin{center}
\begin{tikzpicture}[scale=0.9, every node/.style={font=\small}]
    % --- Bar 1: Harapan ---
    \node[anchor=south west] at (0, 3) {\textbf{1. Jika SEMUA piring utuh}};
    % Bar panjang (Biru)
    \draw[fill=colBlue, draw=black] (0, 2) rectangle (12, 3);
    \node[white, font=\bfseries] at (6, 2.5) {Pendapatan Maksimal = $78 \times \$1.50 = \$117$};

    % --- Bar 2: Kenyataan ---
    \node[anchor=south west] at (0, 0.5) {\textbf{2. Kenyataan yang diterima}};
    % Bar pendek (Oranye)
    \draw[fill=colOrange, draw=black] (0, -0.5) rectangle (7.5, 0.5);
    \node[white, font=\bfseries] at (3.75, 0) {\$73};

    % --- Bar Gap (Selisih) ---
    % Bar Arsiran Merah
    \draw[pattern=north east lines, pattern color=colRed, draw=colRed] (7.5, -0.5) rectangle (12, 0.5);
    \node[colRed, font=\bfseries, align=center] at (9.75, 1) {GAP (Selisih)\\$\$117 - \$73 = \$44$};
    \draw[->, thick, colRed] (9.75, 0.7) -- (9.75, 0.2); % Panah nunjuk ke gap

    % --- Penjelasan Unit Kerugian ---
    \node[anchor=west] at (0, -2.5) {\textbf{3. Kenapa ada GAP \$44?}};
    \node[anchor=west, align=left] at (0, -3.2) {Setiap 1 piring pecah menyebabkan kerugian ganda:};
    
    % Visualisasi 1 Unit Kerugian
    \draw[fill=colBlue!50, draw=black] (0, -4.5) rectangle (2, -3.8);
    \node at (1, -4.15) {\$1.50};
    \node[below, font=\tiny] at (1, -4.5) {Hilang Upah};
    
    \draw[fill=colRed!70, draw=black] (2, -4.5) rectangle (6, -3.8);
    \node[white] at (4, -4.15) {+ Denda \$9.50};
    \node[below, font=\tiny] at (4, -4.5) {Bayar Ganti Rugi};
    
    % Brace Total Rugi per Piring
    \draw[decorate,decoration={brace,amplitude=5pt,mirror,raise=15pt}, thick] (0, -4.5) -- (6, -4.5) 
    node[midway, below=0.8cm] {\textbf{Total Rugi per Piring = \$11}};

\end{tikzpicture}
\end{center}
\vspace{0.5cm}

\textbf{Hitungan:}
Total selisih uang (\$44) disebabkan oleh piring-piring yang pecah. Setiap satu piring pecah mengurangi total sebesar \$11.
\[ \text{Jumlah piring pecah} = \frac{\text{Total Selisih}}{\text{Rugi per Piring}} = \frac{44}{11} = \mathbf{4} \]

\end{document}
